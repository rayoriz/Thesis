Blockchain basierte Distributed Ledgers gehören in der heutigen Zeit zu den größten Forschungsfeldern. Die Gründe liegen hier besonders in den Erfolgen von Kryptowährungen wie Bitcoin, Etherium, usw. Trotz vieler Vorteile sind Blockchain Systeme meist ressourcenintensiv und schwer skalierbar. Daher suchen viele Wissenschaftler nach Alternativen für Blockchain Systeme. Eine dieser Alternativen ist Tangle, welches von der IOTA Stiftung ins Leben gerufen wurde. Tangle ist ein auf gerichteten zyklischen Graphen (Directed Acyclic Graph oder DAG) basierendes Distributed Ledger System, welches Vorteile wie etwa hohe Skalierbarkeit oder Unterstützung für Micropayments fördert. Außerhalb des Kontextes der IOTA Stiftung wurden die Eigenschaften von DAG basierten Distributed Ledger Systemen allerdings noch nicht erforscht. Des Weiteren enthält IOTA bislang noch keine großen Peer-To-Peer Simulationssysteme mit anpassbaren Parametern, welche den Nutzern das Erforschen von wichtigen Charakteristiken bzw. Metriken des Netzwerks und das Vergleichen von verschiedenen Situationen in kontrollierten Bedingungen ermöglicht. Als Lösung dieses Problems schlagen wir CIDDS, ein konfigurierbares und interaktives DAG basiertes Distributed Ledger Simulationssystem (eng. Configurable and Interactive DAG Based Distributed Ledger Simulations Framework), vor . Mit CIDDS können Nutzer eine große skalierbare Simulation mit tausenden von Knoten erstellen und die Eigenschaften der entstandenen DAG Ledgers mit unterschiedlichen Parametern erforschen.